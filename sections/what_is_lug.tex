\section{What is a GNU/Linux user group?}

\subsection{What is GNU/Linux?}

To fully appreciate LUGs' (Linux User Groups') role in the GNU/Linux 
movement, it helps to understand what makes GNU/Linux unique.

GNU/Linux as an operating system is powerful -- but GNU/Linux as an
{\itshape {\bfseries idea}\/} about software development is even more so. GNU/Linux
is a {\bfseries free} operating system: It's licensed under the GNU General
Public Licence (and other open source / free software licences -- though 
proprietary application software is sometimes also included in
particular packagings). Thus, source code is freely available in
perpetuity to anyone. It's maintained by a unstructured group of
programmers world-wide, under technical direction from Linus Torvalds
and other key developers. GNU/Linux as a movement has no central
structure, bureaucracy, or other entity to direct its affairs. While
this situation has advantages, it poses challenges for allocation of
human resources, effective advocacy, public relations, user education,
and training.



(This HOWTO credits the Free Software Foundation's 

\emph{GNU Project} \texttt{\abrurl}
 as the crucial motive force behind creating and furthering a free 
aka open source integrated system.  Thus, it refers to "distributions" 
comprising the GNU operating system atop the Linux kernel as "GNU/Linux".
Yes, the term is awkward, and FSF's request for credit isn't widely 
honoured; but the justice of FSF's claim is obvious.)

(This HOWTO's maintainer is also fully aware that the world at large
will never adopt this usage, justice notwithstanding.  If it seems 
mannered, please indulge him, and respect the gesture.)






\subsection{How is GNU/Linux unique?}

GNU/Linux's loose structure is unlikely to change.  That's a good thing:
It works precisely because people are free to come and go as they
please: {\bfseries Free programmers are happy programmers are effective
programmers}.

However, this loose structure can disorient the new user: Whom
does she call for support, training, or education? How does she know
what GNU/Linux is suitable for?

In part, LUGs provide the answers, which is why LUGs have been vital to
the movement: Because your town, village, or metropolis sports no
Linux Corporation "regional office", the LUG takes on many of the same
roles a regional office does for a large multi-national corporation.

GNU/Linux is unusual in neither having nor being burdened by central
structures or bureaucracies to allocate its resources, train its users,
and support its products. These jobs get done through diverse means: the
Internet, consultants, VARs, support companies, colleges, and
universities. However, increasingly, in many places around the globe,
they are done by a LUG.




\subsection{What is a user group?}

Computer user groups are not new. In
fact, they were central to the personal computer's history:
Microcomputers arose in large part to satisfy demand for affordable,
personal access to computing resources from electronics, ham radio, and
other hobbyist user groups.  Giants like IBM eventually discovered the
PC to be a good and profitable thing, but initial impetus came from the
grassroots, leading to groundbreaking efforts like SHARE (1955-present) 
and DECUS (1961-2008).

In the USA, user groups have changed -- many for the worse --
with the times. The financial woes and dissolution of the largest user
group ever, the Boston Computer Society, were well-reported; but, all
over the USA, most PC user groups have seen memberships decline.
American user groups in their heyday produced newsletters, maintained
shareware and diskette libraries, held meetings and social events, and,
sometimes, even ran electronic bulletin board systems (BBSes). With the
advent of the Internet, however, many services that user groups once
provided migrated to things like CompuServe and the Web.

GNU/Linux's rise, however, coincided with and was intensified by the
general public "discovering" the Internet. As the Internet grew more
popular, so did GNU/Linux: The Internet brought new users,
developers, and vendors. So, the same force that sent traditional user
groups into decline propelled GNU/Linux forward, and inspired new groups 
concerned exclusively with it. 

To give just one indication of how LUGs differ from traditional 
user groups: Traditional groups must closely 
monitor what software users redistribute at meetings.
While illegal copying of restricted proprietary software certainly
occurred, it was officially discouraged -- for good reason.
At LUG meetings, however, that entire mindset simply does not apply:
Far from being forbidden, unrestricted copying of GNU/Linux
should be among a LUG's primary goals.  In fact, there is anecdotal
evidence of traditional user groups having difficulty adapting to
GNU/Linux's ability to be lawfully copied at will.

(Caveat:  A few distributions bundle GNU/Linux with proprietary
software packages whose terms don't permit public redistribution.
Check licence terms, if in doubt.  Offers or requests to copy 
distribution-restricted proprietary software of any sort should be
heavily discouraged anywhere in LUGs, and declared off-topic for all 
GNU/Linux user group on-line forums, for legal reasons.)




\subsection{Avoiding Burnout and Decline}

Since around 2003, LUGs in developed countries have seen a decline
similar to that of traditional user groups.  The causes can be debated,
and might include:

\begin{itemize}
\item GNU/Linux being so successful that it's often perceived as
infrastructure rather than as something new and interesting.
\item LUGs getting lost in the noise of social media.
\item Early adopters critical to making LUGs function moving
on to other interests.
\item 
\emph{Meetup.com} \texttt{\absurl}
, with its strong inward-facing focus, sucking
away available talent and energy, and making LUGs less noticeable.
\item GNU/Linux becoming so much easier to install
and use that focus has shifted to more-specialised topics better served
by more-specialised technical communities (DevOps, bioinformatics,
cloud computing, embedded computing, and many others).
\item LUG leaders poorly managing a generational transition, 
leaving nobody ready to take over as they bow out.
\item Greater ubiquity of the Internet generally, and
specifically reputation-based collaborative sites like 

\emph{StackExchange} \texttt{\abturl}
,

\emph{StackOverflow} \texttt{\abuurl}
, 

\emph{Doctype} \texttt{\abvurl}
, 

\emph{Codeproject} \texttt{\abwurl}
, and 

\emph{Serverfault} \texttt{\abxurl}
, not to mention 
users becoming skilled at Web-searching, making LUGs far less
pragmatically necessary, and the main action involving SaaS sites having 
site scale and network effects with which LUGs cannot compete.
\end{itemize}


A few time-tested tips for averting LUG flameout:

\begin{itemize}
\item Automation is your friend.  Any task that can be scripted, 
should be scripted.
\item Check all your LUG's systems, both technical and social,
for single points of failure (SPoFs).  Keep trying to make sure there are
fallbacks if anything or anyone fails.  Do (and test) backups.  Ensure that nothing
important can be done by only one person.
\item Beware of your LUG, or any individual in it, committing 
to carrying out too much work, or with too great frequency.  It's 
better for a LUG to do less, or have its functions occur less often,
than risk people wearing out and leaving.
\item Remember that if people aren't having fun, they won't 
continue for long.  E.g., if your group becomes less technical and 
more social, don't fret.  It's probably a healthy thing.
\item Carefully guard your significant assets, such as domain
ownership, difficult-to-acquire meeting venues, and the names of key
corporate contacts, and keep them away from problematic people sometimes
drawn to LUGs.  Even if you, say, wrestle your domain away from someone
who's suddenly decided to destroy the LUG (which does happen), the
strife will drive away key people.
\end{itemize}







\subsection{Summary}

For the GNU/Linux movement to grow, among other requirements,
LUGs must proliferate and succeed.  Because of GNU/Linux's
unusual nature, LUGs must provide some of the same functions a "regional
office" provides for large computer corporations like IBM, Microsoft,
and Sun. LUGs can and must train, support, and educate users,
coordinate consultants, advocate GNU/Linux as a computing solution,
and even serve as liaison to local news outlets.



