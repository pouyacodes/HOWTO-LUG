\section{مقدمه}
%\section{Introduction}

\subsection{این راهنما چه هدفی دارد؟}
%\subsection{Purpose}

ای راهنما قصد دارد بعنوان مرجعی برای تاسیس، نگهداری و توسعه گروه های گنو/لینوکس استفاده شود.
تلاش بر این است که علاقمندان با خواندن این راهنما لاگ را به خوبی شناخته و بتوانند به‌درستی در آن
مشارکت کنند. از طرفی این مستند به مدیران و داوطلبان فعال‌تر لاگ یاد می‌دهد که چطور لاگ را به درستی
مدیریت کنند و چه قوانینی را بهتر است در لاگ اعمال کنند.
%The Linux User Group HOWTO is intended to serve as a guide to founding,
%maintaining, and growing a GNU/Linux user group.

گنو/لینوکس یک پیاده سازی بطور آزاد منتشر شده از یونیکس است که برای کامپیوترهای شخصی،
سرورها، کارگاه ها، \lr{PDA} ها و سیستم های نهفته طراحی شده است.
گنو/لینوکس در ابتدا بر روی معماری \lr{i386} توسعه یافت و اکنون طیف گسترده ای از پردازنده‌ها
را از ریز تا درشت پشتیبانی می‌کند.
%GNU/Linux is a freely-distributable implementation of Unix for personal
%computers, servers, workstations, PDAs, and embedded systems.  It was
%developed on the i386 and now supports a huge range of processors from 
%tiny to colossal:

%%%%%%%%%%%%%%%%%%%%%%%%%%%%%%%
% This information does not help readers
%\input{sections/linux_ports}
%%%%%%%%%%%%%%%%%%%%%%%%%%%%%%%

\subsection{منابع بیشتر راجع به لاگ}
%\subsection{Other sources of information}

اگر میخواهید مطالب بیشتری مطالعه کنید،
\fftnt{پروژه مستندسازی لینوکس}{http://www.tldp.org/}
منبع بهتری برای شروع می‌باشد. برای اطلاعات عمومی راجع به گروه های کاربری کامپیوتر نیز لطفا
\fftnt{انجمن گروه‌های کاربران کامپیوتر}{http://apcug2.org/}
را بررسی کنید.

%If you want to learn more, the \emph{Linux Documentation Project} \texttt{http://www.tldp.org/} is a good place to start.
%For general information about computer user groups, please see the \emph{Association of PC Users Groups} \texttt{http://apcug2.org/}.

