

\section{LUG activities}

In the previous section I focused exclusively on what LUGs do and 
should do. This section's focus shifts to practical strategies.

There are, despite permutations of form, two basic things LUGs do:
First, members meet in physical space; second, they communicate
in cyberspace. Nearly everything LUGs do can be seen in terms of
meetings and online resources.




\subsection{Meetings}

As I said above, physical meetings are synonymous with LUGs (and 
most user groups).  LUGs have these kinds of meetings:

\begin{itemize}
\item social
\item technical presentations
\item informal discussion groups
\item online videoconferencing (Jitsi Meet, Zoom, etc.)
\item user group business
\item GNU/Linux installation
\item configuration and bug-squashing
\end{itemize}


What do LUGs do at these meetings?

\begin{itemize}
\item Install distributions for newcomers and strangers.
\item Teach members about GNU/Linux.
\item Compare GNU/Linux to other operating systems.
\item Teach members about application software.
\item Discuss advocacy.
\item Discuss the free software / open-source movement.
\item Discuss user group business.
\item Eat, drink, and be merry.
\end{itemize}







\subsection{Online resources}

The commercial rise of the Internet coincided roughly with that of
GNU/Linux; the latter owes something to the former. The Net has always been
important to development. LUGs are no different: Most have Web
pages, if not whole Web sites. In fact, I'm not sure how else to find a
LUG, but to check the Web.  

It makes sense, then, for a LUG to make use of whatever Internet
technologies they can: Web sites, Jitsi Meet/Zoom/etc., mailing lists, 
wikis, e-mail, Web discussion forums, netnews, etc. As the world of commerce is
discovering, the Net is an effective way to advertise, inform, educate,
and even sell. The other reason LUGs make extensive use of Internet
technology is that the very essence of GNU/Linux is to {\itshape provide\/} a stable and rich platform to deploy these technologies. So,
not only do LUGs benefit from, say, establishment of a Web site,
because it advertises their existence and helps organise members,
but, in deploying these technologies, LUG members 
learn about them and see GNU/Linux at work.

Arguably, a well-maintained Web site is the one must-have, among those
Internet resources.  My essay

\emph{Recipe for a Successful Linux User Group} \texttt{\aduurl}
, for that reason,
spends considerable time discussing Web issues.  Quoting it (in outline form):

\begin{itemize}
\item You need a Web page.
\item Your Web page needs a reasonable URL.
\item You need a regular meeting location.
\item You need a regular meeting time.
\item You need to avoid meeting-time conflicts.
\item You need to make sure that meetings happen as advertised, without fail.
\item You need a core of several enthusiasts.
\item Your core volunteers need out-of-band methods of communication.
\item You need to get on the main lists of LUGs, and keep your entries accurate.
\item You must have login access to maintain your Web pages, as needed.
\item Design your Web page to be forgiving of deferred maintenance.
\item Always include the day of the week, when you cite event dates. Always check that day of the week, first, using cal.
\item Place time-sensitive and key information prominently near the top of your main Web page.
\item Include maps and directions to your events.
\item Emphasise on your main page that your meeting will be free of charge and open to the public (if it is).
\item You'll want to include an RSVP "mailto" hyperlink, on some events.
\item Use referral pages.
\item Make sure every page has a revision date and maintainer link.
\item Check all links, at intervals.
\item You may want to consider establishing a LUG mailing list.
\item You don't need to be in the Internet Service Provider business.
\item Don't go into any other business, either.
\item Walk the walk. (Do the LUG's computing on GNU/Linux.)
\end{itemize}


That essay partly supplements (and partly overlaps) this HOWTO.

Some LUGs using the Internet effectively:

\begin{itemize}
\item 
\emph{Atlanta Linux Enthusiasts} \texttt{\advurl}
\item 
\emph{Boston Linux and Unix} \texttt{\adwurl}
\item 
\emph{Dusseldorfer Linux Users Group} \texttt{\adxurl}
\item 
\emph{India Linux Users Group Delhi} \texttt{\adyurl}
\item 
\emph{Israeli Group of Linux Users} \texttt{\adzurl}
\item 
\emph{Korean Linux Users Group} \texttt{\aeaurl}
\item 
\emph{Linux Mexico (La Cofradia Digital)} \texttt{\aeburl}
\item 
\emph{Linux User Group Austria} \texttt{\aecurl}
\item 
\emph{Linux User Group of Davis} \texttt{\aedurl}
\item 
\emph{Linux User Group of Rochester} \texttt{\aeeurl}
\item 
\emph{Nederlandse Linux Gebruikers Groep (Netherlands Linux Users Group or NLLGG)} \texttt{\aefurl}
\item 
\emph{North Texas Linux Users Group} \texttt{\aegurl}
\item 
\emph{Ottawa Canada Linux Users Group} \texttt{\aehurl}
\item 
\emph{Provence Linux Users Group} \texttt{\aeiurl}
\item 
\emph{Tokyo Linux Users Group} \texttt{\aejurl}
\item 
\emph{Turkish Linux User Group} \texttt{\aekurl}
\end{itemize}




Please let me know if your LUG uses the Internet in an important or
interesting way; I'd like this list to include your group.



