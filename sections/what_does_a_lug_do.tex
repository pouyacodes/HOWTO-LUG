\section{لاگ چه کارهایی انجام می‌دهد؟}
%\section{What does a LUG do?}

به یاد داشته باشید که گنو/لینوکس فاقد بروکراسی و کنترل مرکزی است.
لاگ هم به همین شکل است. اهداف لاگ‌ها به‌اندازه مکان‌هایشان متنوع است
و هیچ طرح و برنامه کلی برای لاگ وجود ندارد. البته که این مستند هم
قرار نیست چنین طرحی را ایجاد کند.
%LUGs' goals are as varied as their locales. There is no LUG master
%plan, nor will this document supply one. Remember: GNU/Linux is free from
%bureaucracy and centralised control; so are LUGs.

اما می‌توان مجموعه ای از اهداف لاگ را مشخص کرد:
%It is possible, however, to identify a core set of goals for a 
%LUG:

\begin{itemize}
\item ترویج \lr{(advocacy)}
%\item advocacy
\item آموزش \lr{(education)}
%\item education
\item پشتیبانی \lr{(support)}
%\item support
\item اجتماعی‌شدن \lr{(socialising)}
%\item socialising
\end{itemize}

هر لاگ ترکیبی از این اهداف و دیگر اهداف منحصر به فرد و مرتبط
به نیاز‌های اعضا می‌باشد.
%Each LUG combines these and other goals uniquely, according to its
%membership's needs.

\subsection{ترویج گنو/لینوکس \lr{(advocacy)}}
%\subsection{GNU/Linux advocacy}


تمایل به ترویج \lr{(advocacy)}
\footnote{
معنای دقیق \lr{advocacy} ترویج نیست و
معادل‌های مختلفی برای ترجمه آن می‌توان استفاده کرد که البته
هیچکدام از آنها نمی‌تواند بطور دقیق معنای آنرا برساند. \lr{advocacy}
در لغت می‌تواند به معنای طرفداری، ترویج، مدافعه و وکالت باشد.
بطور کلی \lr{advocacy} به معنای پشتیبانی از یک ایده می‌باشد.
درطول این مستند معادل‌هایی مانند ترویج یا طرفداری را خواهید دید
و منظور ما از تمام آنها \lr{advocacy} خواهد بود.
}
گنو/لینوکس به‌طور گسترده‌ای احساس می‌شود.
زمانی که شما چیزی را پیدا می‌کنید که ‫به‌خوبی کار می‌کند،
دوست دارید به هرتعداد آدمی که می‌توانید آن را معرفی کنید.
همانطور که نوشتن یک بررسی مثبت از یک روزنامه‌نگار کامپیوتری
راجع به گنو/لینوکس برای این جنبش مفید است، معرفی گنو/لینوکس
به دوستان، همکاران، کارمندان و حتی کارفرمایان هم مفیداست.
%The urge to advocate the use of GNU/Linux is widely felt. When you find
%something that works well, you want to tell as many people as you can.
%While it is certainly beneficial to the movement, each and every time a
%computer journalist writes a positive review of GNU/Linux, it is also
%beneficial every time satisfied GNU/Linux users brief their friends,
%colleagues, employees, or employers.

به‌عنوان کاربر همیشه باید هوشیار باشیم که طوری از گنو/لینوکس
طرفداری کنیم و آن را ترویج دهیم که بازتاب مثبت بر محصول،
سازنده‌ها، توسعه دهنده‌ها و حتی سایر کاربرها داشته باشد.
توجه داشته باشید که طرفداری هم می‌تواند سازنده باشد و هم می‌تواند
صرفا بهانه‌گیری منفی و غیر سازنده باشد.
\fftnt{راهنمای ترویج لینوکس}{http://www.tldp.org/HOWTO/Advocacy.html}
در
\fftnt{پروژه مستندسازی لینوکس}{http://www.tldp.org/}
موجود می‌باشد و پیشنهادهای بسیار کمک کننده ای ارائه می‌دهد.
پروژه دیگری مشابه پروژه مستندسازی لینوکس، به نام 
\ftnt{Linuxmanship}{http://zgp.org/~dmarti/linuxmanship/}
وجود دارد که کار
\tfftnt{دون مارتی}{Don Marti}
می‌باشد.
%There is effective advocacy, and there is ineffective carping: As 
%users, we must be constantly vigilant to advocate GNU/Linux in such a way as
%to reflect positively on the product, its creators and developers, and
%our fellow users. The 
%\emph{Linux Advocacy HOWTO} \texttt{http://www.tldp.org/HOWTO/Advocacy.html}
%, available at the 

%\emph{Linux Documentation Project} \texttt{http://www.tldp.org/}
%, 
%gives some helpful suggestions, as does Don Marti's excellent 

%\emph{Linuxmanship} \texttt{http://zgp.org/~dmarti/linuxmanship/}
% essay.

در طول‌عمر زیاد این مستند گنو/لینوکس تقریبا موفق بوده است و به‌همین دلیل،
نگهدارنده این مستند حجم زیادی از این قسمت مستند را حذف کرده. درواقع
محبوبیت و جایگاه فعلی گنو/لینوکس طوری است که ترویج و طرفداری آن فاقد اهمیت می‌باشد.
%Over the long life of this HOWTO, GNU/Linux more or less won the 
%day, so the HOWTO maintainer has deleted much of this section, and 
%advocacy in his view has become, in his view, overwhelmingly irrelevant.

\subsection{محدودیت‌های ترویج (\lr{advocacy})}
%\subsection{The limits of advocacy}

طرفداری می‌تواند بدون هدف باشد. طرفداری می‌تواند اشتباه و دارای اثر عکس باشد.
طرفداری می‌تواند در ابتدا به‌سادگی نامناسب باشد. این موضوع مستحق تفکر و بحث دقیق است.
زیرا اگر ترویج و طرفداری به اشتباه انجام گیرد می‌تواند تنها به معنای اتلاف وقت باشد.
%Advocacy can be mis-aimed; advocacy can go wrong and be
%counterproductive; advocacy can be simply inappropriate in the first
%place.  The matter merits careful thought, to avoid wasted time or
%worse.

خیلی از تلاش‌ها برای طرفداری به‌طرز وحشتناکی با شکست مواجه می‌شوند
زیرا این طرفداری نتوانسته است به خواسته‌ها و نیاز‌های طرف دیگر داستان
گوش بدهد. (همانطور که \lr{Eric S. Raymond}
\fftnt{می‌گوید}{http://web.archive.org/web/20120131000749/http://www.itworld.com/print/36449}
،
\tfftnt{
به منافع و علاقمندی‌های مشتری خود توجه کنید، نه خودتان.
}{Appeal to the prospect's interests and values, not to yours}
اگر شخصی ستاپ سیستم‌عامل انحصاری که درحال حاضر دارد را می‌پسندد
و آنرا دوست دارد، ترویج تنها وقت شما و او را هدر می‌دهد.
اگر نیازهای او کاملا با \lr{MS-Project}، \lr{MS-Visio} و
\tfftnt{گروه‌افزار اوت‌لوک/استک‌اکسچنج}{Outlook/Exchange groupware}
مطابق است، تلاش برای فروختن چیزی که او واقعا نمی‌خواهد 
تنها باعث آزار او خواهد شد و هیچ‌کس از این رفتار لذت نمی‌برد.
(صرف‌نظر از اینکه آیا واقعا آن نیازها را دارد یا خیر).
تلاش و انرژی خود را برای فرد دیگری که پذیرای آن باشد نگه‌‌دارید.
%Many attempts at advocacy fail ignominiously because the advocate fails
%to listen to what the other party feels she wants or needs.  (As Eric
%S. Raymond 
%\emph{says} \texttt{http://web.archive.org/web/20120131000749/http://www.itworld.com/print/36449}
%, 
%"Appeal to the prospect's interests and values, not to
%yours.") If that person wants exactly the proprietary-OS setup she
%already has, then advocacy wastes your time and hers.  If her
%stated requirements equate exactly to MS-Project, MS-Visio, and
%Outlook/Exchange groupware, then trying to «sell» her what she doesn't
%want will only annoy everyone (regardless of whether her requirements
%list is real or artificial).  Save your effort for someone more
%receptive.  

در این راستا بخاطر بسپارید که برای خیلی از مردم، شاید بیشترشان
یک «مدافع» درواقع یک فروشنده تصور می‌شود. بنابراین طبیعی است
که به‌جای عادلانه گوش‌دادن به شما، درمقابل حرف‌هایتان مقاوت کند.
آنها هرگز کسی را ندیده اند که بدون درنظر داشتن منفعت، رایگان
نرم‌افزاری را دراختیارشان قرار بدهد. از همین رو فکر می‌کنند این طرفداری
به‌هرحال باید سودی برای شما داشته باشد. به‌همین خاطر گوش دادن به صحبت‌های
شما را هم یک لطف شخصی می‌دانند که در حق شما داشتند. چه‌برسه به توصیه‌های
شما هم عمل کنند.
%Along those lines, bear in mind that, for many people, perhaps most, an
%"advocate» is perceived as a salesman, and thus classified as someone to
%resist rather than listen to fairly. They've never heard of someone
%urging them to adopt a piece of software without
%benefiting materially, so they assume there must be something in
%it for you and will push back, and
%act as if they're doing you a personal favour to even listen, let alone
%try your recommendations.  

پیشنهاد می‌کنم فورا بحث را قابل‌لمس و منطقی کنید. به این اشاره‌کنید
که این سیاست‌های نرم‌افزاری در بلندمدت براساس منفعت افراد خواهند بود.
همچنین شخص شما هیچ نفعی از انتخاب ایشان نخواهید برد. این استفاده
مفید تری از زمانتان است تا آنکه با مخاطبانی صحبت کنید که قصد پذیرش
حرف شما را ندارند. بعد از تمام اینها اگر هنوز به این بحث علاقمند بودند،
حداقل با مانعی مصنوعی روبرو نخواهید شد. این موانع مواردی مانند پیش‌داوری
و تفکر غلط مخاطبان راجع به هدف شما از ترویج گنو/لینوکس می‌باشد.
%I recommend bringing such discussions back to Earth
%immediately, by pointing out that software policy should be based in
%one's own long-term self interest, that you have zero personal stake in
%their choices, and that you have better uses for your time than speaking
%to an unreceptive audience. After that, if
%they're still interested, at least you won't face the same artificial
%obstacle.

در همین‌حال مطمئن باشید که درگیر کلیشه‌های طرفداری بی‌مورد از یک سیستم عامل نشده اید.
صرف اعلام نقطه‌نظر خود به یک شخص، بدون دعوت و اینکه او بخواهد صرفا بی‌ادبانه و توهین‌آمیز
تلقی می‌شود. به‌علاوه زمانی‌ که این موضوع به گنو/لینوکس مرتبط است هم این تلاش بی‌فایده خواهد بود.
برخلاف سیستم‌عامل‌های انحصاری، سیستم‌عامل ما قرار نیست بر اساس مقبولیت و انتشار/نگهداری
برنامه‌های سازگارشده خود زنده شود یا بمیرد.
گنو/لینوکس و البته تمام اپلیکیشن‌های مهم آن متن‌باز هستند. جامعه برنامه‌نویسانی که آنها
را نگهداری می‌کنند خودکفا بوده و آن را سالم نگه‌داشته و رشد می‌دهند.
صرف‌نظر از اینکه دنیای بیزینس و عموم مردم عاشق آن هستند و بی‌پروایانه از آن استفاده می‌کنند،
استفاده زیادی دارند و یا حتی اصلا مورد استفاده قرار نخواهند داد.
به‌دلیل قوانین لایسنس متن‌باز آنها، سورس کد به‌طور دائمی در دسترس همگان خواهد بود.
امکان ندارد که گنو/لینوکس بخاطر عدم محبوبیت از سمت یک سری شرکت و کمپانی
«از رده خارج شود». بر این اساس هیچ دلیلی برای پافشاری بر طرفداری از این سیستم‌عامل
وجود ندارد. البته که بعضی از کاربران در انجمن‌هایی این پافشاری و طرفداری بی‌مورد را
انجام می‌دهد که ذکر کردیم. (چطور است اطلاعات را فقط به فردی بدهیم که پذیرای آن است؟
این امر نیاز هرکسی را که شایسته این اطلاعات است برطرف می‌کند.)
%At the same time, make sure you don't live up to the stereotype of the
%OS advocate, either.  Just proclaiming your views at someone without
%invitation is downright rude and offensive. Moreover, when done
%concerning GNU/Linux, it's also pointless:  Unlike the case with proprietary
%OSes, our OS will not live or die by the level of its acceptance and
%release/maintenance of ported applications.  It and all key applications
%are open source: the programmer community that maintains it is
%self-supporting, and would keep it advancing and and healthy regardless
%of whether the business world and general public uses it with wild
%abandon, only a little, or not at all. Because of its open-source
%licence terms, source code is permanently available. GNU/Linux cannot be
%"withdrawn from the market» on account of insufficient popularity, or at
%the whim of some company.  Accordingly, there is simply no point in
%arm-twisting OS advocacy -- unlike that of some OS-user communities we
%could mention.   (Why not just make information available for those
%receptive to it, and stop there?  That meets any reasonable person's
%needs.)

در نهایت درک کنید که
\fftnt{«ارزش استفاده»}{https://en.wikipedia.org/wiki/Use_value}
برای نرم‌افزار برای اکثر مردم غریبه است (مفهوم ارزش‌گذاری نرم‌افزار بر اساس کاربرد آن).
عادت ارزش‌گذاری هرچیز بر اساس
{\itshape هزینه خرید}
قدمت دیرینه دارد.
من در سال ۱۹۹۶ از جوانی در لاگ واقع در برکلی، کالیفرنیا راجع به ریشه‌های
\lr{Caldera Network Desktop}
(نام اولیه توزیع گنو/لینوکسی که توسعه داده بود) در پروژه
سیستم‌عامل دسکتاپ «\lr{Corsair}» توسط شرکت \lr{Novell} شنیدم.
پس از بررسی مدیران اجرایی و فنی آنها متوجه شدند که افسران شرکت‌ها
از هرچه می‌توانند به رایگان دریافت کنند ناراضی هستند. سپس \lr{Caldera}
به آنها راهکاری را با دریافت هزینه ارائه داد.

%Last, understand that the notion of «use value» for software is quite
%foreign to most people -- the notion of measuring software's value by
%what you can do with it.  The habit of valuing everything at
%{\itshape acquisition cost\/} is deeply ingrained.  In 1996, I heard a young
%fellow from Caldera Systems speak at a Berkeley, California LUG about
%the origins of Caldera Network Desktop (the initial name of their GNU/Linux
%distribution) in Novell, Inc.'s «Corsair» desktop-OS project:  In
%surveying corporate CEOs and CTOs, they found corporate officers to be
%inherently unhappy with anything they could get for free.  So, Caldera
%offered them a solution -- by charging money.

از این دیدگاه محتاط بودن درمورد گوشزد کردن هزینه و دشواری‌های توسعه گنو/لینوکس
به مردم به جذب شدن آنها کمک کرده و اعتبار شما را به‌عنوان یک سخنران حفظ می‌کند.
حتی بردن بحث به‌سمت قیمت به معنی قیمت عملکردی می‌تواند بهتر باشد
(مانند ظرفیت ۱۰۰ نفره ایمیل‌های شرکتی با کاربر‌های آفلاین و آنلاین).
در مقابل فهرست کردن نرم‌افزار به سبک خرده‌فروشی با قیمت گذاری به‌اندازه کافی مفید نمی‌باشد.
در نهایت تمام پروژه‌های نرم‌افزاری هزینه ای دارند حتی اگر برچسب قیمت آنها صفر باشد.
هدف اصلی متن‌باز قیمت اولیه نیست. بلکه کنترل بلند مدت روی فناوری اطلاعات و تکنولوژی می‌باشد
(که بخش کلیدی برای یک سازمان است). در یک سیستم انحصاری کاربر (یا کسب‌و‌کار) کنترل فناوری اطلاعات
را از دست داده و در رابطه انحصاری نادرست با فروشنده خود قرار می‌گیرد.
کاربر با اوپن سورس می‌تواند کنترل داشته باشد و کسی نمی‌تواند آزادی را از او بگیرد.
همین موضوع توضیح می‌دهد که چرا مردم (به ویژه مدیران جرایی) درحال حاضر تفاوت بین 
اوپن سورس و کالای انحصاری را (به‌عنوان فرصتی برای کاهش و کنترل ریسک‌های فناوری اطلاعات)
درک می‌کنند. و این امر در بلندمدت بسیار مهم تر از قیمت خواهد بود.
%Seen from this perspective, being conservative about the costs and
%difficulties of GNU/Linux deployments helps make them positively attractive
%-- and protects your credibility as a speaker.  Even better would be
%to frame the discussion of costs in terms of the cost of functionality
%(e.g., 1000-seat Internet-capable company e-mail with offline-user
%capability and webmail) as opposed to listing software as a retail-style
%line-item with pricing:  After all, any software project has costs,
%even if the acquisition price tag is zero, and the real point of open
%source isn't initial cost but rather long-term control over IT -- a key
%part of one's operations:  With proprietary systems, the user (or
%business) has lost control of IT, and is on the wrong side of a monopoly
%relationship with one's vendor.  With open source, the user is in
%control, and nobody can take that away.  Explained that way (as
%opportunity to reduce and control IT risk), people readily understand
%the difference -- especially CEOs -- and it's much more significant over
%the long term than acquisition cost.

\subsection{آموزش گنو/لینوکس}
%\subsection{GNU/Linux education}

لاگ نه‌تنها وظیفه ترویج گنو/لینوکس را دارد، بلکه باید اعضا را نیز آموزش دهد
تا از سیستم‌عامل و دیگر بخش‌های مرتبط به ما استفاده کنند (هدفی که می‌تواند
تغییر زیادی در منطقه محلی فرد ایجاد کند). 
درحالی که دانشگاه‌ها و کالج‌ها به‌طور فزاینده ای گنو/لینوکس را
وارد برنامه تحصیلی خود کرده‌اند، با اینحال باز هم اطلاعات لازم به برخی کاربران نخواهد رسید.
لاگ می‌تواند کمک‌های ابتدایی و پیشرفته ای در تکنولوژی‌های مدیریت سیستم، برنامه‌نویسی
اینترنت و اینترانت و غیره را انجام بدهد.
%Not only is it the business of a LUG to advocate GNU/Linux usage, but
%also to train members, as well as the nearby computing public,
%to use our OS and associated components -- a goal that can make a huge
%real-world difference in one's local area.  While universities and
%colleges are increasingly including GNU/Linux in their curricula, for
%sundry reasons, this won't reach some users.  For those, a LUG can
%give basic or advanced help in system administration, programming,
%Internet and intranet technologies, etc.

برخلاف انتظار بیشتر لاگ‌ها ستون پشتیبانی از شرکت‌ها هستند.
به ازای هر نیرویی که با مشارکت در لاگ مهارت‌های کامپیوتری خود را
افزایش می‌دهد، این مسئولیت از روی دوش شرکت و کمپانی‌ها برداشته خواهد شد.
اگرچه مدیریت یک سیستم گنو/لینوکس خانگی به دقیقا اندازه مدیریت سیستم‌های
بزرگی در مقیاس راه‌اندازی یک
\fftnt{پایگاه داده تحلیلی}{https://en.wikipedia.org/wiki/Data_warehouse}،
\fftnt{مراکز تماس}{https://en.wikipedia.org/wiki/Call_centre}،
یا دیگر تاسیسات دسترسی‌بالا نمی‌رسد. اما کاملا بهتر از تجربه کار با
مایکروسافت ویندوز خواهد بود.
با پیشرفت لینوکس به‌سمت
\fftnt{فایل سیستم‌های ژورنالی}{https://en.wikipedia.org/wiki/Journaling_file_system}،
\tfftnt{دسترسی سطح بالا}{High availability}،
افزونه‌های بلادرنگ، و دیگر قابلیت‌های سطح‌بالا لینوکس،
مرز مبهم بین گنو/لینوکس و یونیکس‌های واقعی کاملا ناپدید شده اند.
%In an ironic twist, many LUGs have turned out to be a backbone of
%corporate support: Every worker expanding her computer skills through
%LUG participation is one fewer the company must train.  Though home
%GNU/Linux administration doesn't exactly scale to running corporate data
%warehouses, call centres, or similar high-availability facilities, it's
%light years better preparation than MS-Windows experience.  As Linux has
%advanced into journaling filesystems, high availability, real-time
%extensions, and other high-end Unix features, the already blurry line
%between GNU/Linux and «real» Unixes vanished entirely.

نه‌تنها چنین آموزش‌هایی نوعی آموزش به کارمندان است، بلکه با افزایش
اهمیت فناوری تکنولوژی برای اقتصاد جهانی، خدماتی به جامعه نیز ارائه می‌کند.
برای مثال در کلان‌شهر‌های آمریکا لاگ‌ها گنو/لینوکس را به مدارس، کسب‌و‌کار‌های کوچک،
جامعه،  ارگان‌های عمومی و سایر محیط‌های غیر شرکتی نیز آورده اند.
این امر باعث می‌شود اهداف مربوط به ترویج و همچنان آموزش همگانی برای مردم براورده شود.
با افزایش بیشتر حضور شرکت‌ها در اینترنت و نیاز به فراهم کردن دسترسی برای پرسنل خود،
یا دیگر عملکرد‌های مربوط به گنو/لینوکس، لاگ‌ها فرصت‌هایی برای مشارکت جمعی خود را کسب می‌کنند.
این مشارکت از طریق تلاش‌هایی برای آگاه سازی و آموزش انجام می‌گیرد.
این اتفاق روح سخاوتمند ذات جامعه گنو/لینوکس و نرم‌افزار‌ آزاد/متن‌باز را به انجمن منتقل می‌کند.
اکثر کاربرها نمی‌توانند مانند لینوس توروالدز برنامه‌نویسی کنند اما همه ما می‌توانیم تلاش و زمان خود را
برای کاربران دیگر، جامعه گنو/لینوکس و البته محیط گسترده‌تر اطراف خود صرف کنیم.
%Not only is such education a form of worker training, but it will also
%serve, as information technology becomes increasingly vital to the
%global economy, as community service: In the USA's metropolitan areas,
%for example, LUGs have taken GNU/Linux into local schools, small businesses,
%community and social organisations, and other non-corporate
%environments. This accomplishes the goal of advocacy and also
%educates the general public.  As more such organisations seek Internet
%presence, provide their personnel dial-in access, or other
%GNU/Linux-relevant functions, LUGs gain opportunities for community
%participation, through awareness and education efforts -- extending to
%the community the same generous spirit characteristic of GNU/Linux and the
%free software / open source community from its very beginning. Most
%users can't program like Torvalds, but we can all give time and
%effort to other users, the GNU/Linux community, and the broader
%surrounding community.

گنو/لینوکس برای این سازمان‌ها انتخاب مناسبی است. زیرا استقرارها
آنها را متعهد به پروانه‌ها، بروزرسانی‌ها و یا نگهداری‌های گران قیمت
نخواهد کرد. اگر فنی، خوش سلیقه و اقتصادی باشید، گنو/لینوکس همچنان بر روی
سخت افزار‌هایی که از طرف شرکت‌های تولید سخت افزار رها شده اند نیز ‫به‌خوبی
کار می‌کند. معمولا سازمان‌های غیرانتفایی/غیرانتفاعی مایل به استفاده از این
سخت‌افزار‌های ترد شده هستند. درواقع یک سیستم با مشخصات پنتیوم ۲ هسته ای
که در کمد اتاقتان وجود دارد، می‌تواند «واقعا کار کند» اگر کسی روی آن
گنو/لینوکس نصب کند.

%GNU/Linux is a natural fit for these organisations, because deployments
%don't commit them to expensive licence, upgrade, or maintenance fees.
%Being technically elegant and economical, it also runs very well on
%cast-off corporate hardware that non-profit organisations are only too
%happy to use: The unused Pentium Core 2 in the closet can do {\bfseries real
%work}, if someone installs GNU/Linux on it.

علاوه بر این آموزش در طول زمان به تحقق اهداف دیگر لاگ نیز کمک خواهد کرد.
مخصوصا حمایت و پشتیبانی. آموزش بهتر به معنای حمایت بهتر است.
این امر باعث سهولت در آموزش و رشد آسان تر انجمن خواهد شد.
بنابراین آموزش سنگ‌بنای تمام تلاش‌ها را شکل می‌دهد.
اگر فقط ۲ یا ۳ درصد جمعیت لاگ، بار دیگر اعضای لاگ را به‌دوش بکشد،
رشد آن لاگ متوقف خواهد شد. مطمئن باشید
{\bfseries {\itshape
اگر اعضای جدید و بی‌تجربه کمک‌های لازم را از لاگ دریافت نکنند، مدت زیادی مشارکت نخواهد کرد.
}}
اما اگر درصد بیشتر اعضای لاگ دیگر اعضا را حمایت کنند، لاگ با چنین محدودیت‌هایی روبرو نخواهد شد.
آموزش (و البته به همان اندازه حمایت برای پروژه‌هایی مانند وب‌سرور آپاچی،
 \lr{X.org, Freedesktop.org, Tex, LaTeX}
 غیره) کلید این تکاپو خواهد بود. آموزش کاربران جدید را به متخصصان آینده تبدیل می‌کند.
%In addition, education assists other LUG goals over time, in
%particular that of support: Better education means better
%support, which in turn facilitates education, and eases the 
%community's growth.  Thus, education forms the entire effort's keystone:
%If only two or three percent of a LUG assume the remainder's support
%burden, that LUG's growth will be stifled. One thing you can count on:
%{\bfseries {\itshape If new and inexperienced users don't get needed help
%from their LUG, they won't participate there for long\/}}.
%If a larger percentage of members support the rest, the LUG will not
%face that limitation. education -- and, equally, support for
%allied projects such as the Apache Web server, X.org, Freedesktop.org, 
%TeX, LaTeX, etc.  -- is key to this dynamic: Education turns new users into
%experienced ones.

نهایتا گنو/لینوکس یک محیط عملیاتی خود-مستندساز است.
به معنای دیگر، نوشتن و انتشار مستندات انجمن ما به خودمان مربوط است.
از همین رو اطمینان حاصل کنید که ممبر‌های لاگ
\tfftnt{پروژه مستند سازی لینوکس}{Linux Documentation Project}
و میرورهای آن را می‌شناسند.
به راه‌اندازی یک سایت میرور از \lr{LDP} نیز فکر کنید.
مطمئن شوید که هر مستندی که می‌تواند به رشد لاگ کمک کند را
به‌صورت عمومی منتشر می‌کنید. این مستندات شامل ارائه‌های فنی، آموزش‌ها،
پرسش و پاسخ‌های محلی و غیره باشند. برای عمومی کردن این مستندها می‌توانید
از طریق
{\ttfamily comp.os.linux.announce}, \lr{LDP}
و دیگر منابع مربوطه اقدام کنید.
معمولا مستندات لاگ نمی‌توانند در سطح جهانی سود برسانند. دلیل این اتفاق
تنها عدم اطلاع رسانی صحیح می‌باشد. اجازه ندهید این اتفاق بیوفتد.
بسیار محتمل است که فردی در لاگ سوال یا مشکلی با چیزی داشته باشد
و شخصی در جای دیگری همان سوال یا مشکل را در سر داشته باشد.
 
%Finally, GNU/Linux is a self-documenting operating environment: In other
%words, writing and publicising our community's documentation is up to
%us.  Therefore, make sure LUG members know of the 
%\emph{Linux Documentation Project} \texttt{\acourl}
% and its worldwide mirrors.
%Consider operating an LDP mirror site.  Also, make sure to
%publicise -- through {\ttfamily comp.os.linux.announce}, the LDP, and other
%pertinent sources of information -- any relevant documentation the LUG
%develops: technical presentations, tutorials, local FAQs, etc.  LUGs'
%documentation often fails to benefit the worldwide community for no
%better reason than not notifying the outside world.  Don't let that
%happen:  It is highly probable that if someone at one LUG had a question
%or problem with something, then others elsewhere will have it, too.

\subsection{پشتیبانی از گنو/لینوکس \lr{(support)}}
%\subsection{GNU/Linux support}

البته که برای
{\bfseries تازه‌کاران}
اصلی ترین نقش لاگ پشتیبانی از گنو/لینوکس می‌باشد.
البته نباید پشتیبانی را صرفا حمایت و پشتیبانی
{\itshape فنی}
دانست. پشتیبانی معنایی فراتر از صرف پشتیبانی فنی دارد.
%Of course, for the {\bfseries newcomer}, the primary role of a
%LUG is GNU/Linux support -- but it is a mistake to suppose that 
%support means only {\itshape technical\/} support for new users. It
%should mean much more.

لاگ‌ها این فرصت دارند که از موارد فوق پشتیبانی کنند:
%LUGs have the opportunity to support:

\begin{itemize}
\item
کاربران
%\item users
\item \tfftnt{مشاوران}{consultants}
%\item consultants
\item
مدارس، کسب‌وکارها و سازمان‌های غیرانتفاعی
%\item businesses, non-profit organisations, and schools
\item
جنبش گنو/لینوکس
%\item the GNU/Linux movement
\end{itemize}

\subsubsection{کاربران}
%\subsubsection{Users}

شایع ترین شکایت کاربران جدید زمانی که برای اولین بار
گنو/لینوکس نصب می‌کنند، شیب تند یادگیری آن است.
البته این ذات تمام یونیکس‌های مدرن می‌باشد.
(این عبارت در سال ۱۹۹۷ وقتی که اولین نگهدارنده این
مستند آن را می‌نوشت صحیح می‌بود. اما خوشبختانه درحال‌حاضر
کمتر صحت دارد و دیگر یادگیری گنو/لینوکس به آن اندازه سخت نیست.)
بهرحال این منحنی یادگیری، قدرت و انعطاف‌پذیری
یک سیستم‌عامل واقعی را نیز به همراه دارد.
معمولا لاگ منبع اصلی یک کاربر جدید برای هموارسازی
این منحنی یادگیری می‌باشد.
%New users' most frequent complaint, once they have GNU/Linux
%installed, is the steep learning curve characteristic of all modern 
%Unixes. (That sentence was true in 1997 when this HOWTO's first
%maintainer wrote it, but happily very little, any more.)  With that learning
%curve, however, comes the power and flexibility of a real operating
%system. A LUG is often the new user's main resource, to flatten the
%learning curve.

گنو/لینوکس در دهه اول خود بهترین منابع مطبوعاتی را بدست آورد
که نباید در ذکر آنها اهمال کاری کرد.
مهم ترین و با قدمت ترین مجله‌ماهانه (باقیمانده) آمریکا
\fftnt{روزنامه لینوکس}{http://www.linuxjournal.com/}
است. به‌تازگی مجلات دیگری هم پا به عرصه نهاده‌اند.
\fftnt{فرمت لینوکس}{http://www.linuxformat.com}‌(انگلستان)،
\fftnt{مجله لینوکس}{http://www.linux-magazine.com/}
(شرکت انتشارات آلمانی که به زبان‌های انگلیسی،
آلمانی، لهستانی، پرتغالی برزیلی، اسپانیایی
مجلات را منتشر می‌کند. نسخه آمریکای شمالی آن
با نام
\tfftnt{مجله حرفه‌ای لینوکس}{Linux Pro Magazine}
شناخته می‌شود.)،
\tfftnt{متن‌باز برای تو}{https://www.opensourceforu.com}
(هندی و سابقا با نام
\tfftnt{لینوکس برای تو}{LINUX For You}
شناخته می‌شد.)،
\fftnt{دایره کامل}{http://fullcirclemagazine.org/}
(بین‌المللی بوده و توزیع‌های خانواده اوبونتو را پوشش می‌دهد.)،
\fftnt{صدای لینوکس}{http://linuxvoice.com/}(انگلستان)،
\fftnt{لینوکس آسان}{http://www.easylinux.de/}(آلمان)،
\fftnt{لینوکس یوزر}{https://www.linux-user.de/}(آلمان)،
\fftnt{اوبونتو یوزر}{http://www.ubuntu-user.com/}(شرکت انتشاراتی آلمان واقع در انگلیس)
و
\fftnt{مجله نرم‌افزار آزاد}{http://freesoftwaremagazine.com/}({\itshape صدای متن‌باز} گذشته)

%During GNU/Linux's first decade, it gained some first-class journalistic 
%resources, which should not be neglected:  The main (surviving) monthly 
%magazine of longest standing is {\itshape 
%\emph{Linux Journal} \texttt{http://www.linuxjournal.com/}
%\/} (USA).
%More recently, they've been joined by 
%{\itshape \emph{Linux Format} \texttt{http://www.linuxformat.com}
%\/} (UK),
%{\itshape 
%\emph{Linux Magazine} \texttt{http://www.linux-magazine.com/}
%\/} (German publishing firm; publishes in English, German, Polish, Brazilian Portuguese, and Spanish; North American edition is named {\itshape Linux Pro Magazine\/}),
%{\itshape 
%\emph{Open Source For You} \texttt{https://www.opensourceforu.com}
%\/} (India; formerly {\itshape LINUX For You)\/},
%{\itshape \emph{Full Circle} \texttt{http://fullcirclemagazine.org/}
%\/} (international; covers Ubuntu family distributions), 
%{\itshape \emph{Linux Voice} \texttt{http://linuxvoice.com/} \/} (UK), 
%{\itshape \emph{easyLinux} \texttt{http://www.easylinux.de/}\/} (German),
%{\itshape \emph{LinuxUser} \texttt{https://www.linux-user.de/}\/} (German), 
%{\itshape \emph{Ubuntu User} \texttt{http://www.ubuntu-user.com/}\/} (German publishing firm; in English), 
%{\itshape \emph{Free Software Magazine} \texttt{http://freesoftwaremagazine.com/}
%\/} (formerly {\itshape The Open Voice\/})

مجلات آنلاین برجسته و سایت‌های خبری با چرخه‌های هفته‌ای یا
انتشارات بهتر شامل
\ftnt{Linux Weekly News}{http://lwn.net/}،
\ftnt{DistroWatch Weekly}{http://distrowatch.com/weekly.php}،
\ftnt{Linux Today}{http://linuxtoday.com}
و
\ftnt{FreshNews}{http://www.freshnews.org/}
می‌باشند.
%Standout on-line magazines and news sites with weekly or better publication 
%cycles include {\itshape \emph{Linux Weekly News} \texttt{http://lwn.net/} \/}, 
%{\itshape \emph{DistroWatch Weekly} \texttt{http://distrowatch.com/weekly.php} \/},
%{\itshape \emph{Linux Today} \texttt{http://linuxtoday.com}\/}, and
%\emph{FreshNews} \texttt{http://www.freshnews.org/}.

تمام این منابع کار لاگ‌ها را برای منتشر کردن اخبار و اطلاعات
ضروری آسان کرده اند. این منابع راجع به باگ‌فیکس‌ها،
مشکلات امنیتی، پچ‌ها، کرنل‌های جدید و غیره می‌باشند.
اما کاربران جدید نیز باید از این موارد آگاه باشند
و یادبگیرند که همیشه جدید ترین کرنل‌ها از
\emph{kernel.org}
در دسترس خواهند بود. مانند کرنل لینوکس،
\fftnt{پروژه مستندسازی لینوکس}{http://www.tldp.org/}
نیز جدید ترین نسخه‌های \lr{HOWTO}‌ها را برخلاف
توزیع‌های گنو/لینوکس برپایه \lr{DVD/CD} و غیره دارد.
%All of these resources have eased LUGs' job of spreading essential
%news and information --  about bug fixes, security problems, patches, 
%new kernels, etc., but new users must still be made aware of
%them, and taught that the newest kernels are always
%available from 
%\emph{kernel.org} \texttt{https://www.kernel.org/},
%that the 
%\emph{Linux Documentation Project} \texttt{http://www.tldp.org/}
% has newer versions of Linux HOWTOs than do DVD/CD-based GNU/Linux
%distributions, and so on.
کاربران میان‌رده و حرفه‌ای نیز از گسترش به‌موقع و مفید
نکات، حقایق و رازها مطلع می‌شوند.
به‌دلیل جنبه‌های متنوع و تنوع دنیای گنو/لینوکس حتی
کاربران پیشرفته هم گاها ترفند یا تکنیک‌های جدیدی را
به‌سادگی با مشارکت در لاگ یاد خواهند گرفت.
بعضی وقت‌ها بسته‌های نرم‌افزاری را یاد خواهند گرفت که
هرگز اسمشان را هم نشنیده بودند. بعضی اوقات مجموعه دستورات
رازآلودی از {\ttfamily vi} را به‌یاد خواهند آورد که از دوران
دانشگاه از آن استفاده نکرده بودند.
%Intermediate and advanced users also benefit from proliferation of
%timely and useful tips, facts, and secrets. Because of the GNU/Linux
%world's manifold aspects, even advanced users often learn new tricks or
%techniques simply by participating in a LUG. Sometimes, they learn of
%software packages they didn't know existed; sometimes, they just
%remember arcane {\ttfamily vi} command sequences they've not used since
%college.

\subsubsection{مشاوران \lr{(Consultants)}}
%\subsubsection{Consultants}

لاگ‌ها با ایجاد انجمنی که همه می‌توانند درکنار یکدیگر باشند
به مشاوران
(\tftnt{consultant}{Professional advisor})
 و مشتری‌هایشان کمک می‌کنند که همدیگر را پیدا کنند.
مشاوران نیز با اشتراک تجربه و راهبری لاگ‌ متقابلا به لاگ سود می‌رسانند.
کاربران جدید و بی‌تجربه هم از لاگ و هم از مشاوران سود می‌برند.
با اینکه درخواست حمایت‌های معمول و ساده کاربران تازه‌کار توسط لاگ‌ها
رایگان اجابت می‌شود، اما مشکلات و نیازهای پیچیده‌تر آنها (که نیاز
به پرداخت هزینه دارد) می‌تواند به‌واسطه مشاورانی که در لاگ می‌توان پیدا
کرد برطرف شود.
%LUGs can help consultants find their customers and vice-versa,
%by providing a forum where they can come together.
%Consultants also aid LUGs by providing experienced leadership.
%New and inexperienced users gain benefit from both LUGs and
%consultants, since their routine or simple requests for support are
%handled by LUGs {\itshape gratis\/}, while their complex needs and problems --
%the kind requiring paid services -- can be fielded by consultants found 
%through the LUG.

مرز اینکه یک درخواست نیازمند کمک گرفتن از مشاوران است یا خیر گاهی
نامشخص است. اما اکثر مواقع تفاوت آشکار است.
درحالی‌که لاگ نمی‌خواهد به ارجاع دادن کاربران جدید به مشاوران مشهور
شود (که بی احترامی و بسیار در تضاد با گنو/لینوکس می‌باشد) همچنین
دلیلی هم وجود ندارد که لاگ به وساطت بین کاربرانی که نیاز به مشاور دارند
و متخصصینی که این مشاوره حرفه ای را ارائه می‌دهند کمک نکند.
%The line between support requests needing a consultant and those
%that don't is sometimes indistinct; but, in most cases, the difference
%is clear. While a LUG doesn't want to gain the reputation for
%pawning new users off unnecessarily on consultants -- as this is simply
%rude and very anti-GNU/Linux behaviour -- there is no reason for LUGs not to
%help broker contacts between users needing consulting services and
%professionals offering them.

\begin{caveat}
این تفاوت برای انسان‌های باهوش و خیرخواه واضح است.
اما همیشه افرادی هم وجود دارند کار را برای حمایت رایگان
از کاربران جدید سخت می‌کنند و عمدا راجع به این محدودیت‌هایی
که خود ایجاد کرده اند احمقانه عمل می‌کنند. نکته ای که قبل تر
به آن اشاره کردم را به یاد بیاورید. بخش زیادی از مردم
هر چیز را بر اساس قیمت‌گذاری آن (به‌جای ارزش استفاده آن)
ارزش و اعتبار می‌دهند.
{\itshape
این شامل مواردی که به رایگان دریافت می‌کنند هم خواهد بود.
}
این موضوع باعث می‌شود برخی به ویژه در دنیای کسب‌و‌کار بی‌حساب و شدیدا
از پشتیبانی فنی لاگ استفاده (و سواستفاده) کنند.
این افراد درکنار استفاده‌های شان از لاگ، شدیدا به همه‌چیز هم اعتراض
دارند و از اطلاعات ناکافی، سرعت کم، سوالات تکراری کاربر‌ها و
هر چیزی که برایشان مطلوب نیست شکایت دارند. به عبارتی دیگر،
آنها طوری رفتار می‌کنند که انگار داوطلبان لاگ موظف هستند در قبال
هزینه ای که دریافت نکرده اند به مانند یک فروشنده خدمات بدهند.
این درحالی است که مانند هزینه‌ای که برای لاگ پرداخت کرده اند،
هیچ احترامی به آن نمی‌گذارند.
\end{caveat}
%Caveat:  While «the difference is clear» to intelligent people of goodwill,
%the Inevitable Ones are {\itshape also\/} always with us, who act willfully
%dense about the limits of free support when they have pushed those
%limits too far.  Remember, too, my earlier point about the vast majority
%of the population valuing everything at acquisition cost (instead of use
%value), {\itshape including what they receive for free\/}.  This leads some,
%especially some in the corporate world, to use (and abuse) LUG
%technical support with wild abandon, while simultaneously complaining
%bitterly of its inadequate detail, insufficient promptness, supposedly
%unfair expectations that the user learn and not re-ask minor variations on
%the same question endlessly, etc.  In other words, they treat relations
%with LUG volunteers the way they would a paid support vendor, but one
%they treat with {\itshape zero respect\/} because of its zero acquisition
%cost.

ضرب‌المثلی در دنیای مشاوره وجود دارد که راجع به اعمال «فیش درمانی"
برای چنین برخورد‌هایی می‌گوید: بخاطر نظام ارزشی که بالاتر به آن اشاره شد،
{\bfseries
اگر به مشاوره شما بی‌اعتنایی شد یا به‌درستی استفاده نشد، کافیست
مبلغ بیشتری را بابت آن دریافت کنید.
}
\LTRfootnote{if your consulting advice is poorly heeded and poorly used,
it just might be the case that you need to charge more}
در مقابل انجمن‌های فنی با نظام ارزشی کاملا متفاوت
ذاتا از
\tfftnt{"فرهنگ هدیه"}{Gift culture}
پیروی می‌کنند.
اعضا با افزایش شهرت خود درمیان همتایان خود به مقام و اعتبار
دست پیدا می‌کنند که آن را از طریق مشارکت قابل مشاهده بدست می‌آورند.
از جمله این مشارکت‌ها ارائه کد، مستندات و کمک‌های فنی به عموم جامعه می‌باشد.
%In the consulting world, there's a saying about applying «invoice therapy"
%to such behaviour:  Because of the value system alluded to above, if
%your consulting advice is poorly heeded and poorly used, it just might
%be the case that you need to charge more.  By contrast, the technical
%community has often been characterised as a «gift culture", with a
%radically different value system: Members gain status through enhanced
%reputation among peers, which in turn they improve through visible
%participation:  code, documentation, technical assistance to the public,
%etc.

درگیری بین دو فرهنگ ارزش محور بسیار متفاوت، اجتناب‌ناپذیر است
و می‌تواند اندکی زشت باشد.
فعالان لاگ باید قبل از بریدن سر یک تازه‌کار وساطت کنند
و مودبانه پیشنهاد بدهند که درصورت نیاز به خدمات بهتر
و البته دارای هزینه (برپایه مشاوره) متوسل شوند.
آدم‌ها همیشه قضاوت می‌کنند. این مرز (که درخواست افراد
نیاز به هزینه کردن به یک مشاور دارد یا می‌توان آن را در
لاگ انجام داد) ذاتا جای بحث دارند.
%Clash between the two very different value-based cultures is inevitable
%and can become a bit ugly.  LUG activists should be prepared to intercede
%before the ingrate newcomer is handed her head on a platter, and
%politely suggest that her needs would be better served by paid
%(consultant-based) services. There will always be judgement calls;
%the borderline is inherently debatable and a likely source of
%controversy.

در زیر چند علامت لیست شده که نشان می‌دهد پرسشگر باید به
کمک‌های برپایه مشاوره ارجاع داده شود:
%Telltale signs that a questioner may need to be transitioned to consulting-based assistance include:

\begin{itemize}
\item
پافشاری بر گرفتن راهکار به شکل «دستورالعمل» بدون نیاز
به یاد گرفتن اساس و بنیاد تکنولوژی.
%\item An insistence on getting solutions in «recipe» (rote) form,
%with the apparent aim of not needing to learn technological 
%fundamentals.

\item
بارها پرسیدن فقط یک سوال یکسان (یا قبلا پاسخ داده شده).
%\item Asking the same questions (or ones closely related) repeatedly.

\item
پافشاری بر روی کمک گرفتن در {\itshape خصوصی} و به‌صورت فردی
آن‌هم از کسانی که در انجمن‌های {\itshape عمومی} فعال هستند
(مانند احتماع گنو/لینوکس).
%\item Insisting on {\itshape private\/} assistance from helpers active in
%{\itshape public\/} (GNU/Linux community) forums.

\item
فراهم کردن توضیحات مبهم یا بی‌ثبات که در طی زمان تغییر می‌کنند
برای مشکل خود.
%\item Providing only vague problem descriptions, or ones that change with time.

\item
قطع کردن جواب به‌منظور پرسیدن سوالات بیشتر
(نشان‌دهنده عدم توجه به پاسخ‌ها).
%\item Interrupting answers in order to ask additional questions
%(suggesting lack of attention to the answers).

\item
تقاضا برای پاسخ بهتر یا ارائه سریع تر جواب (نشان می‌دهد
که زمان و مشکل سوال‌کننده ارزشمند است، اما زمان و پاسخ
کمک‌کننده ارزشی ندارد).
%\item Demanding that answers be recast or delivered more quickly
%(suggesting that the questioner's time and trouble are
%valuable, but that helpers' are not).

\item
پرسیدن سوالات غیرمعمول پیچیده، زمان‌بر و یا سوالات چندبخشی.
%\item Asking unusually complex, time-consuming, and/or multipart
%questions.
\end{itemize}

درکل، اعضای لاگ از کمک کردن به افراد خاصی بیشتر خوشحال می‌شوند.
البته این در ذات داوطلبی کار آنها می‌باشد. اعضایی که بنظر می‌آید
در «فرهنگ هدیه» بیشتر مشارکت خواهند کرد. آنها این فرهنگ
را یاد گرفته و بعدا زمانی که نوبت‌شان فرا رسید با درس دادن همان فرهنگ
به اعضای جدید لاگ آن را جاودان خواهند کرد.
فراهم کردن کمک‌های مالی برای این اعضای آینده‌ساز می‌تواند ایده خوبی باشد
و با نیازهایشان مطابقت داشته باشد.اما مشخصا اولیت‌
و ارزش‌های دیگر نیز وجود خواهند داشت.
%In general, LUG members are especially delighted to help, on a volunteer
%basis, members who seem likely to participate in the «gift
%culture» by picking up its body of lore and, in turn, perpetuating it
%by teaching others in their turn. Certainly, there's nothing wrong with
%having other priorities and values, but such folk may in some cases be
%best referred to paid assistance, as a better fit for their needs.
طبق بررسی‌های دیگری که می‌تواند مفید باشد یا حتی نباشد،
کارهایی وجود دارند که ممکن است شخصی مایل باشد آنها را
رایگان انجام بدهد تا صرفا به جامعه کمک کرده باشد،
و در مقابل شخص دیگر حتی درازای دریافت هزینه هم
از انجام آن صرف‌نظر کند.
یک همکار کامپیوتری که ناگهان تبدیل به مشتری شما می‌شود،
بسیار فرد متفاوتی را به شما نشان خواهد داد. مسئولیت‌های
آن فرد کاملا تغییر خواهد کرد و حتی بیشتر خواهد شد.
این نشان می‌دهد که گاها اگر کمک‌ها داوطلبانه نباشند،
نوع ارتباط و رابطه کاملا تغییر خواهد کرد.
%An additional observation that may or may not be useful, at this point:
%There are things one may be willing to do for free, to assist others in the
%community, that one will refuse to do for money:  Shifting from
%assisting someone as a volunteer fundamentally changes the relationship.
%A fellow computerist who suddenly becomes a customer is a very different
%person; one's responsibilities are quite different, and greater.  You're
%advised to be aware, if not wary, of this distinction.

\subsubsection{مدارس، کسب‌وکارها و سازمان‌های غیرانتفاعی}
%\subsubsection{Businesses, non-profit organisations, and schools}

لاگ‌ها همچنان فرصت دارند از کسب‌و‌کارها و سازمان‌‌های محلی
پشتیبانی کنند. این پشتیبانی دو جنبه دارد:
۱) لاگ‌ها می‌توانند کسب‌وکار و سازمان‌هایی که می‌خواهند
سیستم‌عامل (و اپلیکیشن‌های) ما در بخشی از تلاش‌های
کامپیوتری و فناوری‌اطلاعات شان قراردهند را پشتیبانی کنند.
۲) لاگ‌ها می‌توانند کسب‌وکارها و سازمان‌هایی که درحال
توسعه نرم‌افزار برای گنو/لینوکس هستند را پشتیبانی کنند،
به نیاز کاربران آنها پاسخ دهند، به نصب یا پشتیبانی از
توزیع‌های آنها کمک کنند و غیره.
%LUGs also have the opportunity to support local businesses and
%organisations. This support has two aspects: First, LUGs can support
%businesses and organisations wanting to use our OS (and its 
%applications) as a part of their
%computing and IT efforts. Second, LUGs can support local businesses
%and organisations developing software for GNU/Linux, cater to users,
%support or install distributions, etc.

پشتیبانی ای که لاگ‌ها می‌توانند به شرکت‌هایی که از گنو/لینوکس
به‌عنوان بخشی از سیستم خود استفاده می‌کنند تنها اندکی با کمکی
که اشخاصی که در خانه گنو/لینوکس را امتحان می‌کنند تفاوت دارد.
برای مثال کامپایل کردن کرنل لینوکس واقعا تفاوتی ندارد.
گرچه گاها پشتیبانی کردن از کسب‌وکار‌ها نیازمند پشتیبانی از
نرم‌افزارهای انحصاری می‌باشد. برای مثال دیتابیس‌های
\lr{Oracle, Sybase} و \lr{DB2}
(و یا
\lr{VMware, CrossOver Linux}
و موارد مشابه).
برخی از متخصصان لاگ در این حوزه‌ها ممکن است به کسب‌وکار‌ها کمک کنند
تا به استفاده از گنو/لینوکس رو بیاورند.
%The support LUGs can provide to local businesses wanting to use GNU/Linux as
%a part of their computing operations differs little from the help LUGs
%give individuals trying GNU/Linux at home. For example, compiling the Linux
%kernel doesn't really differ. Supporting businesses, however, may
%require supporting proprietary software -- e.g., the Oracle, Sybase,
%and DB2 databases (or VMware, CrossOver Linux, and such things).
%Some LUG expertise in these areas may help businesses make the leap
%into GNU/Linux deployments.

این ما را به‌سمت اولین نوع از پشتیبانی که لاگ می‌تواند به
کسب‌وکار‌های محلی بدهد می‌کشاند: لاگ‌ها می‌توانند مانند یک
"\fftnt{شرکت تسویه وجوه}{https://en.wikipedia.org/wiki/Clearing_house_(finance)}"
برای اطلاعات موجود در چند مکان دیگر عمل کنند. برای مثال:
%This leads us directly to the second kind of support a LUG can give to
%local businesses: LUGs can serve as a clearinghouse for information
%available in few other places. For example:

\begin{itemize}
\item
کدام \lr{ISP} محلی به لینوکس علاقمند است؟
%\item Which local ISP is Linux-friendly?
\item
کدام فروشنده سخت‌افزاری \lr{PC} لینوکس می‌سازد؟
%\item Are there any local hardware vendors building Linux PCs?
\end{itemize}


نگهداری و عمومی‌سازی این نوع اطلاعات نه‌تنها به اعضای لاگ کمکی می‌کند،
بلکه به کسب‌وکارهای علاقمند هم کمک می‌کند و آنها را به ادامه استفاده از
گنو/لینوکس تشویق می‌کند.
این حتی می‌تواند در برخی موارد به ایجاد محیطی رقابتی کمک کند که
در آن کسب‌وکار‌ها تشویق می‌شوند که مجموعه را دنبال کنند.
%Maintaining and making this kind of information public not only helps
%the LUG members, but also helps friendly businesses and encourages
%them to continue to be GNU/Linux-friendly. It may even, in some cases, help
%further a competitive environment in which other businesses are
%encouraged to follow suit.

\subsubsection{توسعه نرم‌افزار آزاد/متن‌باز}
%\subsubsection{Free / open-source software development}

نهایتا لاگ‌ها می‌توانند با تقاضا و سازمان‌دهی کمک‌های خیریه
به جنبش گنو/لینوکس کمک کنند.
\fftnt{کریس براون}{mailto:cbbrowne@cbbrowne.com}
بیش از هرکسی که می‌شناسم به این موضوع فکر کرده و او مشارکت
های زیر را انجام داده است:
%Finally, LUGs may also support the movement by soliciting and
%organising charitable giving. 
%\emph{Chris Browne} \texttt{mailto:cbbrowne@cbbrowne.com}
% has thought about this issue as much as
%anyone I know, and he contributes the following:

\paragraph{کریس براون در انسان‌دوستی نرم‌افزار آزاد/متن‌باز}
%\paragraph{Chris Browne on free software / open source philanthropy}

مشارکت بیشتر می‌تواند سازمان‌هایی که به گنو/لینوکس ربط دارند را
برای اسپانسر مالی شدن تشویق کند. با وجود میلیون‌ها کاربر، منطقی
بنظر می‌رسد که کاربران قدردان هرکدام مشارکتی انجام دهند
(کاربر قدرشناس به ازای ۱۰۰ دلاری که برای سیستم عامل مایکروسافت
نپرداخته است، می‌تواند در گنو/لینوکس مشارکت کند).
داشتن میلیون‌ها کاربر و هزاران دلار «قدردانی» به ازای هر یک از آنها
که جمعا می‌تواند به
{\itshape هزاران میلیون دلار}
برسد منجر به توسعه ابزار‌ها و اپلیکیشن‌های بهتر گنو/لینوکس خواهند شد.

%A further involvement can be to encourage sponsorship of various
%GNU/Linux-related organisations in a financial way.  With the
%multiple millions of users, it would be entirely plausible for grateful
%users to individually contribute a little. Given millions of users, and 
%the not-unreasonable sum of a hundred dollars of «gratitude» per user (\$100 being
%roughly the sum {\itshape not\/} spent this year upgrading a Microsoft OS),
%that could add up to {\itshape hundreds of millions\/} of dollars towards
%development of improved GNU/Linux tools and applications.

یک گروه کاربری می‌تواند اعضا را به مشارکت در پروژه‌های مختلف تشویق کند.
داشتن نوعی «معافیت مالیاتی خیریه» می‌تواند اعضا را برای مشارکت مستقیم در گروه،
دریافت کسر مالیاتی مناسب تشویق کند. این درحالی است که این مشارکت‌ها
به سازمان‌های دیگر نیز کمک خواهد کرد.
%A user group can encourage members to contribute to various
%"development projects". Having some form of «charitable tax exemption"
%status can encourage members to contribute directly to the group,
%getting tax deductions as appropriate, with contributions flowing on to
%other organisations.

اهداف و پروژه‌های وجود دارند که اعضای لاگ در آرزوی پشتیبانی و حمایت از آنها هستند.
تشویق کردن اعضای لاگ برای انجام دادن این حمایت بسیار شایان می‌باشد.
%It is appropriate, in any case, to encourage LUG members to direct
%contributions to organisations with projects and goals they
%individually wish to support.

این بخش نامزد‌های ممکن را فهرست می‌کند. اولیتی در آنها وجود ندارد و این فهرست
صرفا باعث می‌شود گزینه‌هایی را بررسی کرده باشید. اکثر آنها در آمریکا به‌عنوان
خیریه ثبت شده اند و به همین سبب مشارکت‌های آمریکا را معاف از مالیت می‌کند.
%This section lists possible candidates. None is being explicitly 
%recommended here, but the list represents useful food for
%thought.  Many are registered as charities in the USA, thus
%making US contributions tax-deductible.

اینجا سازمان‌هایی با فعالیت‌های ویژه در راستای توسعه نرم‌افزاری‌هایی که
بر روی گنو/لینوکس کار می‌کنند فهرست شده است:
%Here are organisations with activities particularly directed towards
%development of software working with GNU/Linux:

\begin{itemize}
\item
\fftnt{بنیاد لینوکس}{http://www.linuxfoundation.org/about}
%\emph{The Linux Foundation} \texttt{http://www.linuxfoundation.org/aboutadhurl}

\item
\fftnt{دبیان/نرم‌افزار در منافع عمومی}{http://www.debian.org/donations.html}
%\emph{Debian / Software In the Public Interest} \texttt{http://www.debian.org/donations.html}

\item
\fftnt{بنیاد نرم‌افزار آزاد}{https://my.fsf.org/associate/support_freedom}
%\emph{Free Software Foundation} \texttt{https://my.fsf.org/associate/support_freedom}

\item
\fftnt{پروژه کی‌دی‌ئی}{http://www.kde.org/community/donations/}
%\emph{KDE Project (KDE e.V.)} \texttt{http://www.kde.org/community/donations/}

\item
\fftnt{بنیاد گنوم}{http://www.gnome.org/friends/}
%\emph{GNOME Foundation} \texttt{http://www.gnome.org/friends/}

\item
\fftnt{حفاظت از آزادی نرم‌افزار}{https:/sfconservancy.org/}
%\emph{Software Freedom Conservancy} \texttt{https:/sfconservancy.org/}

\item 
\fftnt{بنیاد موزیلا}{https://foundation.mozilla.org/en/}
%\emph{The Mozilla Foundation} \texttt{https://foundation.mozilla.org/en/}
\end{itemize}

مشارکت در این سازمان‌ها تاثیر مستقیمی بر حمایت و پشتیبانی از تولید
نرم‌افزارهایی با قابلیت توزیع آزاد که در گنو/لینوکس استفاده می‌شوند دارد.
به ازای هر دلار چنین مشارکتی قطعا بیش از سرمایه‌گذاری‌های دیگر برای انجمن نفع دارد.
%Contributions to these organisations have the direct effect of
%supporting creation of freely redistributable software usable with
%GNU/Linux.  Dollar for dollar, such contributions almost certainly yield
%greater benefit to the community than any other kind of spending.

سازمان‌های دیگری هم وجود دارند که ارتباط مستقیم کمتری با گنو/لینوکس دارند.
اما باوجود کمک کردن به آنها هم ارزشمند می‌باشد. فهرست آنها در زیر قابل مشاهده می‌باشد:
%There are also organisations less directly associated with GNU/Linux, that
%may nonetheless be worthy of assistance, such as:

\begin{itemize}
\item
\fftnt{بنیاد مرزهای الکترونیکی}{http://www.eff.org/}

که در سانفرانسیسکو می‌باشد، به اختصار \lr{EFF} یک
\fftnt{سازمان عضو محور}{https://en.wikipedia.org/wiki/Membership_organization}
بوده که بر اساس حمایت اهداکنندگان برای محافظت از حقوق
اساسی ما کار می‌کند. این سازمان صرف‌نظر از تکنولوژی، به
مطبوعات، سیاست‌گذاران، عموم مردم راجع به مسائل آزادی‌های مدنی
مرتبط به تکنولوژی و به‌عنوان مدافع آن آزادی‌ها آموزش می‌دهد.
در میان فعالیت‌های مختلف ما، \lr{EFF} از ما حمایت می‌کند.
درمقابل قوانین غلط می‌ایستد. به حفظ حقوق افراد پرونده‌های
قضایی را آماده کرده و از آنها دفاع می‌کند. کمپین‌های جهانی
را راه‌اندازی می‌کند. پروپوزال و مقالات در لبه پیشرو تکنولوژی
را معرفی می‌کند. مکررا میزبان رویدادهای آموزشی می‌باشد.
مرتبا با مطبوعات در ارتباط بوده و آرشیو اطلاعات آزادی مدنی
دیجیتال را در یکی از پر ارجاع ترین وبسایت‌های دنیا منتشر می‌کند.





%\item The 
%\emph{Electronic Frontier Foundation} \texttt{http://www.eff.org/}

%Based in San Francisco, EFF is a donor-supported membership organization
%working to protect our fundamental rights regardless of technology; to
%educate the press, policy-makers, and the general public about civil
%liberties issues related to technology; and to act as a defender of 
%those liberties.
%Among our various activities, EFF opposes misguided
%legislation, initiates and defends court cases preserving individuals'
%rights, launches global public campaigns, introduces leading edge
%proposals and papers, hosts frequent educational events, engages the
%press regularly, and publishes a comprehensive archive of digital civil
%liberties information at one of the most linked-to Web sites in the
%world.

\item
فاند پروژه لاتک۳

\fftnt{گروه کاربران تک (\lr{TUG})}{http://www.tug.org/}
بر روی «نسل بعدی» نسخه سیستم انتشارات لاتک کار می‌کنند. گنو/لینوکس
یکی از پلتفرم‌هایی است که تک و لاتک بهترین پشتیبانی را از آن می‌کند.


%\item The LaTeX3 Project Fund

%The 
%\emph{TeX Users Group (TUG)} \texttt{http://www.tug.org/}
% is
%working on the «next generation» version of the LaTeX publishing
%system, known as LaTeX3.  GNU/Linux is one of the platforms on which TeX
%and LaTeX are best supported.

برای انجام کمک‌های مالی می‌توانید از مشخصات زیر استفاده کنید:
% Donations for the project can be sent to:
\begin{latin}
\begin{tscreen}
\begin{verbatim}
TeX Users Group
c/o Robin Laakso, executive director
TeX Users Group
PO Box 2311
Portland, OR 97208-2311
\end{verbatim}
\end{tscreen}
\end{latin}

و یا به‌جای آن به‌صورت آنلاین به
\fftnt{صحفه کمک‌های مالی انجن تاگ}{https://www.tug.org/donate.html}
مراجعه کنید.
%Alternatively, donations can be made 
%\emph{online} \texttt{https://www.tug.org/donate.html}.

\item
\fftnt{پروژه گوتنبرگ}{http://www.gutenberg.org/}

هدف پروژه گوتنبرگ این است که موتون کتاب‌های عمومی را به‌طور آزاد و رایگان
در دسترس همگان قرار دهد. این مستقیما به لینوکس ارتباطی ندارد، اما عادلانه
بنظر می‌رسد که ارزشش را دارد. آنها به‌طور فعال استقلال پلتفرم را ترویج داده
و تشویق می‌کنند و این یعنی «محصولات آنها» بسیار در گنو/لینوکس قابل استفاده می‌باشد.

%\emph{Project Gutenberg} \texttt{http://www.gutenberg.org/}

%Project Gutenberg's purpose is to make freely available in electronic
%form the texts of public-domain books.  This isn't directly a «Linux
%thing", but seems fairly worthy, and they actively encourage platform
%independence, which means their «products» are quite usable with GNU/Linux.

\item
\fftnt{پروژه رونبرگ}{http://runeberg.org/}

پروژه رونبرگ مانند پروژه گوتنبرگ است. با این تفاوت که بر روی
ارائه ویرایش‌های
\tfftnt{ادبیات کلاسیک اسکاندیناوی}{classic Nordic (Scandinavian) literature}
به‌طور آزاد در اینترنت تمرکز دارد.

%\emph{Project Runeberg} \texttt{http://runeberg.org/}
% 
%Project Runeberg is similar to Project Gutenberg, except concentrating
%on making editions of classic Nordic (Scandinavian) literature openly
%available over the Internet.

\item
\fftnt{بنیاد آموزش متن‌باز}{http://www.osef.org/donations.html}



%\emph{Open Source Education Foundation} \texttt{http://www.osef.org/donations.html}

هدف بنیاد آموزش متن‌باز بهبود آموزش مدارس (از پیش‌دبستانی تا انتهای ۱۲ام)
از طریق استفاده‌کردن از تکنولوژی‌ها و مفاهیمی که از متن‌باز و جنبش نرم‌افزار آزاد
آورده شده اند می‌باشد.دو پروژه \lr{Tux4Kids} و \lr{OSEF} در کنار هم یک توزیع گنو/لینوکس
(\lr{Knoppix}
برای کودکان) را بر پایه توزیع
\tfftnt{کلاوس ناپر}{Klaus Knopper}
که
\tfftnt{ناپیکس}{Knoppix}
نام داشت ساختند که هدف آن کودکان، والدین، معلمان و دیگر صاحب منصبان مدرسه بود.
\lr{OSEF}
سیستم‌عامل را در آزمایشگاه‌های کامپیوتر مدرسه نصب و پشتیبانی می‌کرد و
\tfftnt{بسته آموزشی برای دانش‌آموزان}{K-12 Box}
را به‌عنوان یک بسته
\tfftnt{کامپیوتر ورک استیشن پلاگ و پلی}{Plug and Play workstation computer}
برای آزمایشگاه‌های کامپیوتر دانش‌آموزان توسعه داد.

%The Open Source Education Foundation's purpose to enhance K-12 education
%through the use of technologies and concepts derived from The Open
%Source and Free Software movement.  In conjunction with Tux4Kids, OSEF 
%created a bootable distribution of GNU/Linux (Knoppix for Kids) based 
%on Klaus Knopper's Knoppix, aimed at kids, parents, teachers, and 
%other school officials. OSEF installs and supports school computer labs, 
%and has developed a «K12 Box» as a compact Plug and Play workstation 
%computer for student computer labs.
\end{itemize}

(توجه داشته باشید که اضافه کردن پیشنهادات اضافه به خیریه‌های
مرتبط به گنو/لینوکس در این مستند مورد استقبال قرار می‌گیرد.)
%(Please note that suggested additions to the above list of GNU/Linux-relevant 
%charities are most welcome.)

\subsubsection{جنبش گنو/لینوکس}
%\subsubsection{Linux movement}

من در این مستند دقیقا گفتم که به چه چیزی 
{\bfseries جنبش گنو/لینوکس}
می‌گویم. این بهترین راه برای توصیف پدیده جهانی گنو/لینوکس است:
گنو/لینوکس بروکراسی نیست، اما سازمان‌دهی شده است. این یک شرکت نیست،
اما برای کسب‌وکارها در همه‌جا اهمیت دارد. بهترین پشتیبانی ای که لاگ
می‌تواند از جنبش گنو/لینوکس انجام دهد این است که انجمن‌های محلی را قدرتمند،
پرجنب‌وجوش و روبه‌رشد نگه‌دارد.
گنو/لینوکس به‌صورت بین‌المللی توسعه یافته است. این موضوع را به‌سادگی
با دیدن فایل \lr{MAINTAINERS} در سورس کد کرنل لینوکس می‌توانید متوجه بشوید.
گنو/لینوکس همچنان به‌صورت بین‌المللی مورد استفاده قرار می‌گیرد.
این افزایش روزافزون
\tfftnt{پایگاه کاربری}{User base}
کلید موفقیت‌های مداوم گنو/لینوکس می‌باشد و برای همین است که لاگ‌ها بسیار مهم هستند.

%I have referred throughout this HOWTO to what I call the {\bfseries GNU/Linux
%movement}. There really is no better way to describe the
%international GNU/Linux phenomenon: It isn't a bureaucracy, but is
%organised. It isn't a corporation, but is important to businesses
%everywhere. The best way for a LUG to support the international GNU/Linux
%movement is to keep the local community robust, vibrant, and
%growing. GNU/Linux is {\itshape developed\/} internationally, which is easy
%enough to see by reading the kernel source code's 
%\url{MAINTAINERS} file -- but
%GNU/Linux is also {\itshape used\/} internationally.  This ever-expanding
%user base is key to GNU/Linux's continued success, and is where the LUGs
%are vital.

قدرت این جنبش در سطح بین‌المللی در ارائه قدرت کامپیوتری بی‌سابقه
به ازای هزینه و آزادی نهفته است. راز این موفقیت در استقلال
از کنترل انحصار و مالکیت است.هر زمان که یک فرد، گروه، کسب‌وکار
و یا سازمانی ارزش ذاتی گنو/لینوکس را تجربه می‌کند،  جنبش رشد می‌کند.
لاگ‌ها به وقوع این اتفاق کمک می‌کنند.
%The movement's strength internationally lies in offering
%unprecedented computing power and sophistication for its cost and
%freedom. The keys are value and independence from proprietary control.
%Every time a new person, group, business, or organisation experiences
%GNU/Linux's inherent value, the movement grows.  LUGs help that
%happen.

\subsection{اجتماعی‌سازی لینوکس}
%\subsection{Linux socialising}

آخرین هدف لاگ را که اجتماعی‌سازی لینوکس است پوشش خواهیم داد.
از جهاتی این سخت ترین هدف برای گفتگو است. زیرا مشخص نیست که
چند لاگ و به چه اندازه به این فعالیت می‌پردازند.
درحالی که بسیار نادر است که لاگ با اهداف مذکور دیگر درگیر نباشد،
ممکن است لاگ‌هایی وجود داشته باشند که عامل اجتماعی‌سازی در آنها مهم نباشد.
%The last goal of a LUG we'll cover is socialising -- in some ways,
%the most difficult goal to discuss, because it isn't clear how
%many or to what degree LUGs do it. While it would be strange to
%have a LUG that didn't engage in the other goals, there may be
%LUGs for which socialising isn't a factor.

بنظر می‌رسد وقتی چند کاربر گنو/لینوکس به‌هم می‌رسند، خوش می‌گذرانن، جشن و پایکوبی
درست می‌کنند و خیلی وقت‌ها آبجو می‌نوشند. لینوس توروالدز همیشه یک هدف ثابت برای
لینوکس داشت: لذت بیشتر بردن.
برای هکرها، توسعه‌دهنده‌های کرنل و کاربران گنو/لینوکس هیچ چیز مانند دانلود کردن
یک کرنل جدید،
\tfftnt{ری‌کامپایل}{recompiling}
کردن یک کرنل قدیمی، دست‌وپنجه نرم کردن با یک
\tfftnt{مدیر پنجره}{window manager}
یا یک محیط دسکتاپ، هک کردن یک سری کد یا تجربه کردن یک کامپیوتر امبدد لینوکس نیست.
\footnote{مترجم هنگام خواندن این پاراگراف بسیار تحت تاثیر قرار گرفته و شدیدا تائید می‌کند.}
لذت خالص گنو/لینوکس لاگ‌های بسیاری را درکنار هم نگه داشته است و سبب می‌شود که لاگ‌ها
به‌طور طبیعی اجتماعی‌سازی شوند.

%It seems, however, that whenever two or three GNU/Linux users get together,
%fun, hijinks, and, often, beer follow. Linus Torvalds has
%always had one enduring goal for Linux: to have more fun. For hackers,
%kernel developers, and GNU/Linux users, there's nothing quite like
%downloading a new kernel, recompiling an old one, fooling with a
%window manager or desktop environment, hacking some code, or experimenting
%with an innovative embedded Linux computer. GNU/Linux's sheer fun keeps many
%LUGs together, and leads LUGs naturally to socialising.

منظور من از «اجتماعی‌سازی» در اینجا بیشتر به معنی اشتراک‌گذاری تجربه،
ایجاد دوستی، تحسین و احترام متقابل است. معانی دیگری نیز وجود دارد
(دانشمندان علوم اجتماعی آن را
\fftnt{فرهنگ‌سازی}{https://en.wikipedia.org/wiki/Acculturation}
هم معنی می‌کنند).
فرهنگ‌سازی شما را از «یکی از آنها» به «یکی از ما» تبدیل می‌کند.
در هر جنبش، موسسه و یا انجمنی نیاز به یک فرایند و الگو از اتفاقات وجود دارد
تا چنین چیزی رقم بخورد. در فرهنگ گنو/لینوکس می‌گوییم تازه‌کار به یک هکر تبدیل شده است.

%By «socialising", here I mean primarily sharing experiences, forming
%friendships, and mutually-shared admiration and respect. There is
%another meaning, however -- one social scientists call
%{\itshape acculturation\/}. In any movement, institution, or human
%community, there is the need for some process or pattern of events in
%and by which, to put it in GNU/Linux terms, newcomers are turned into
%hackers. In other words, acculturation turns you from «one of them» to
%"one of us".

بسیار حائز اهمیت است که کاربران جدید به یادگیری فرهنگ، مفاهیم، سنت‌ها
و واژه‌گان گنو/لینوکس رو بیاورند. فرهنگ‌سازی گنو/لینوکس برخلاف فرهنگ‌سازی «دنیای واقعی"
می‌تواند در
\tfftnt{لیست پستی}{mailing list}،
انجمن‌های تحت وب، یوزنت انجام بگیرد. گرچه مورد آخر به دلیل فرهنگ ضعیف کاربران آن
و اسپم‌ها اثربخشی کمتری دارد.
لاگ‌ها در این کار معمولا از لیست‌های پستی، انجمن‌های تحت وب یا نیوگروپ‌ها بسیار کارآمدتر
و بهتر هستند. دلیل این برتری این است که لاگ‌ها تعامل بیشتری دارند و بیشتر بر روی
افراد تمرکز دارند.
%It is important that new users come to learn GNU/Linux culture,
%concepts, traditions, and vocabulary.  GNU/Linux acculturation, unlike «real
%world» acculturation, can occur on mailing lists, Web forums, and
%Usenet, although the latter's efficacy is challenged by poorly
%acculturated users and by spam. LUGs are often much more efficient at
%this task than are mailing lists, Web forums, or newsgroups, precisely
%because of LUGs' greater interactivity and personal focus.

